Concurrent systems have been of interest for decades. With their simple but expressive semantics, concurrent systems become a good choice to fit the data and analyze the underlying mechanics. However, learning and analyzing such concurrent systems are computationally difficult. When dealing with big data sets, the techniques state-of-the-art appear to be insufficient, either in term of efficiency or in term of precision.
In this thesis, we propose a refined modeling framework ABAN (Asynchronous Binary Automata Network) and develop reachability analysis techniques based on ABAN: PermReach (Reachability via Permutation search) and ASPReach (Reachability via Answer Set Programming). Then we propose two model learning methods: CRAC (Completion via Reachability And Correlations) and M2RIT (Model Revision via Reachability and Interpretation Transitions) using respectively continuous and discrete data to fit the model and using reachability properties to constrain the output models.
Chapter 1 states briefly the background and the contribution of our research.  Chapter 2 introduces the state of the art on modeling frameworks, model checkers, different update schemes of modelings and model learning techniques. Some of them are referenced in the following chapters.
Chapter 3 presents our modeling framework and its relating reachability analyzers based on static analysis. We focus on the inconclusive cases of pure static analysis and extract the key components preventing from a direct solution. We then apply heuristics on these components, solving them with a limited search to reach a more conclusive result of the reachability problem.
Chapter 4 presents the methodology of model learning.
Our model learners CRAC and M2RIT perform in fact a model selection.  They choose a model from the candidates satisfying all the provided reachability constraints. However the number of candidate models can be exponential, our model revisers can shrink the search space with constraints when generating the models.
Chapter 5 shows some comparative and exploratory tests and their results on the methods presented in Chapter 3 and Chapter 4. PermReach and ASPReach are more efficient than traditional model checkers on the reachability analysis and they perform a more conclusive analysis while holding the running time in the same scale as pure static analyzers have.
Chapter 6 concludes the thesis and proposes some possible future work.