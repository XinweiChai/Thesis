\chapter{Introduction}
\section{Context and Motivations}
In the studies of concurrent systems, modeling is an inevitable topic.
The modeling frameworks discussed in this thesis are all designed for biological use but they can be potentially useful in other domains, \textit{e.g.} robotics, human engineering.

Models helps one to access, analyze and control the real system. 
It is a tool to help people to understand the interaction of the components in real systems and the integral behavior of the systems.

A good model is a model which

\begin{itemize}
    \item is consistent with the corresponding real system
    
    The model reproduces certain important behaviors.
    In the ideal situation, the model bisimulates the real system.
    \item is observable
    
    To allow one to verify the behaviors, the state (historical, current and future) and the mechanics of the model has to be observable.
    
    \item allows one to access the I/O of the model
    
    With full control of the I/O, we can carry some unfeasible tests in real system.
    %\item allows one to access the state (historical, current and future) of the model
    \item has related analyzers of various properties
    
    Some properties are not verifiable via finite enumeration.
    \item can be translated from/to other models 
\end{itemize}

Normally, the above metrics are self-constrained:
The finer the model is, the bigger the computational complexity is (simulation, verification, \textit{etc.}).
In this thesis, we will focus on modelings, their related analyzers of system properties and model revision based on these properties.

\subsection{Models in Computational Biology}
Bioinformaticians are interested in highly abstracted models because they need abstract representation and/or flexibility to make model compatible with unknown biological knowledge.

Tractability with big data is also important.
``Big'' can refer to two meanings: one is that biological systems can be huge, with enormous components and interactions in between; another is that the number and the size of data sets can be huge.

To model the real system, we need components to represent genes, RNA messengers, proteins, metabolites, \textit{etc}.
At this stage, we can carry out a first-step abstraction.
The synthesis of proteins is under the instruction of RNA messengers which are synthesized according to genes.
This correspondence allows us to compress three entities into one.
Their inner behaviors (\textit{e.g.} protein phosphorylation, activation/inhibition of genes) can be reflected by additional values or parameters.

\section{Previous Work of the Team}
Process Hitting \cite{pauleve2012}, Asynchronous Automata Network \cite{folschette2015}, Learning from Interpretation Transitions \cite{ribeiro2015learning}

\section{Contributions}
The main contributions of the thesis are the following:
\begin{itemize}
    \item Reachability analyzers
    \item Model revisers
\end{itemize}


A story to tell, how I developed all this work
\section{Organization of the Manuscript}
The summary of each section