\chapter{Refined Reachability Analysis}
\begin{mybox}
Several modeling frameworks and model checking techniques are introduced in Chapter \ref{chap:stateOfTheArt}.
We noticed that even though there exist already exact model checkers and static analyzers for reachability problems, they are not sufficient.
Exact model checkers face always the state space explosion problem when analyzing big models (of about 50 variables);
Static analyzer PINT, designed for Process Hitting/Automata Network is however, theoretically inconclusive, i.e. not able to provide a global solution.

This chapter is going to formally clarify, modify the definitions of the modelings in order to make them more suited for our own reachability analyzers, then explain how their related static analysis functions.
Also, to deal with the inconclusiveness problem, we propose at first doing some preprocessings to simplify the topology of the networks, then apply two new approaches (PermReach and ASPReach) based on over-approximation, covering the shortcomings of pure static analysis.
They try to recover the lost information due to abstraction done by static analysis, constructing a more close approximation of the real dynamics in order to gain a better conclusiveness.
\end{mybox}

\section{Background}
Reachability problem on formal models is a critical challenge where both validation problems (whether the model satisfies the \textit{a priori} knowledge) and prediction problems (properties to be discovered) meet. 
From a formal point of view, numerous biological properties in computational models can be transformed to reachability properties. 
For example, the reachability of state 0/1 of a could represent the activation/inhibition of certain gene or synthesis of a protein, while initial state could represent initial observation in an experiment.
If the reachability of a certain state contradicts with the \textit{a priori} knowledge, one can modify the model and/or design a new experiment to verify whether there is an error in the \textit{a priori} knowledge or former observation. 
Also, reachability analysis is of help to medicine design: for example if one wants to prevent the carcinogenesis of a cell (target state), one possible solution is to find the critical pathways towards the target state and design a medicine to cut them in order to keep the cell healthy.

To tackle the complexity issue, symbolic model checking \cite{burch1992symbolic} based on ordered binary decision diagrams (OBDDs) and SAT-solvers (satisfiability) \cite{abdulla2000symbolic} have been studied over years, but still fail to analyze big biological systems with more than $1000$ variables. 
Bounded Model Checking (BMC) \cite{clarke2001bounded} is an efficient approach but generally not complete as its searching depth is limited to a given integer $k$.

Beside these approaches, abstraction is an efficient strategy to deal with such models of big scale. 
It aims at approximating the model while keeping the most important parts influencing the reachability.
Abstract methods often have better time-memory performance but with a loss of information. 
They solve usually a simplified version of the original model, i.e. the results from these approaches are not necessarily compatible with all the properties of the original model.
While studying reachability problems, the system dynamics is abstracted to static causalities between states and transitions.

\section{Asynchronous Binary Automata Network}
In \cite{folschette2015}, Paulev\'e \textit{et al.} have proposed Asynchronous Automata Network (AAN) to model concurrent systems and Local causality graph (LCG) \cite{pauleve2017reduction,folschette2015,pauleve2011} to analyze the reachability of AAN.
This interpretation drastically reduces the searching state-space thus avoids costly global search \cite{pauleve2012}. 
However, this pure static analysis is not complete as there are inconclusive cases which can not be decided reachable or not.
LCG can only conclude with the following two constraints:
\begin{itemize}
    \item With no cycles (Section \ref{sec:cycles})
    \item With no \textbf{AND gates} (Section \ref{sec:conclusiveness})
\end{itemize}
We are thus going to refine the reachability analysis to deal with more instances.
To attack the inconclusiveness problem, we have designed a new discrete modeling framework for concurrent systems \cite{chai2018heuristic}: Asynchronous Binary Automata Network (ABAN).
In ABAN, we adapted LCG to SLCG (Simplified LCG) to address reachability problem.
This approach refers to a static abstraction of the reachability (with an over-approximation of the real dynamics).
Under ABAN, SLCG is more prone to conclude than LCG under AAN (Section \ref{sec:SLCG}).

\subsection{Definitions}
\begin{definition}[ABAN]
An ABAN is a triplet $AB = (\Sigma,L,T)$, where:
\begin{itemize}
\item $\Sigma\triangleq\{a,b,\ldots\}$ is the finite set of automata with every component having a Boolean state;
\item $LS\triangleq \underset{a\in \Sigma}{\cup} \{a_0,a_1\}$ is the set of all local states, $L\triangleq \underset{a\in \Sigma'}{\times} \{a_0,a_1\}$ is the set of joint states where $\Sigma'\subseteq\Sigma$. Particularly, if $\Sigma'=\Sigma$, $L$ is the set of global states. 
\item $T\triangleq \{A\rightarrow b_i\mid b\in \Sigma \land A\in L\}$ is the set of transitions, where $A$ (called head) is the set of required state(s) for transition $tr=A\to b_i$ , which allows to flip $b_{1-i}$ to $b_i$ (called body). In other words, transition $tr$ is said fireable iff $A\subseteq s$, where $s$ is the current global state.
\end{itemize}
\end{definition}

A local states represent the state of one automaton, e.g. $a_1$, while a global state represents the joint state of all automata, e.g. $\langle a_0, b_1,c_0 \rangle$ where $L=\{a,b,c\}$.

\begin{definition}[Dynamics]
    From current global state $s$, the global state after firing transition $tr=A\to b_j$ is denoted $s \cdot tr = s \setminus \{b_i\} \cup \{b_j\}, b_i \in s$.
    The state of a certain automaton $a$ is noted $(s\cdot tr)[a]$.
\end{definition}

To describe the evolution in an ABAN, we use the notion of trajectory.
\begin{definition}[Trajectory]
Given an ABAN $AB = (\Sigma,L,T)$ and a global initial state $\alpha\in L$, a trajectory $t$ from $\alpha$ is a sequence of transitions $t=tr_1::\cdots :: tr_i::\cdots ::tr_n$ with $tr_i\in T$ and each $tr_i$ is fireable in $(s \cdot tr_1 \cdot \ldots \cdot tr_{i-1})$.
From $\alpha$, the global state after firing all transitions of $t$ is $(s \cdot tr_1 \cdot \ldots \cdot tr_n)$, denoted $s \cdot t$.
\end{definition}

\begin{definition}[State sequence]
Given an ABAN $AB = (\Sigma,L,T)$ and a global initial state $\alpha\in L$ and trajectory $t$, the state sequence $seq=s_1::\cdots :: s_i::\cdots ::s_n$ with $s_i\in LS$ is formed by the updated local states during the trajectory $t$.
\end{definition}

\begin{example}
    Figure \ref{exampleABAN} illustrates an ABAN with set of transitions $T=\{\acm{b_1,c_1}{a_0}{a_1},\acm{e_1}{a_0}{a_1},\acm{d_0}{b_0}{b_1},\acm{d_1}{c_0}{c_1},\acm{b_1}{d_0}{d_1}\}$ and initial state $\alpha=\langle a_0,b_0,c_0,d_0,e_0\rangle$ and a possible trajectory is $t=\acm{d_0}{b_0}{b_1}::\acm{b_1}{d_0}{d_1}::\acm{d_1}{c_0}{c_1}::\acm{b_1,c_1}{a_0}{a_1}$.
    After firing the transitions in trajectory $t$, the state becomes $\Omega=s\cdot t=\langle a_1,b_1,c_1,d_1,e_0\rangle$, and $\omega= a_1= (\alpha\cdot t)[a]$. The state sequence is $seq=b_1::d_1::c_1::a_1$.
\end{example}

\begin{figure}[ht]
\centering
\begin{tikzpicture}[apdotsimple/.style={apdot},scale=0.7, every node/.style={scale=0.8}]
\scriptsize
\TSort{(0,0)}{a}{2}{l}
\TSort{(2,0)}{b}{2}{l}
\TSort{(4,0)}{c}{2}{l}
\TSort{(6,0)}{d}{2}{l}
\TSort{(8,0)}{e}{2}{l}

% with delays
\path[local transitions]

	(a_0) edge node[auto] {\{$b_1, c_1$\}} (a_1)
    (a_0) edge[bend right] node[right] {$\{e_1\}$} (a_1)
	(b_0) edge node[auto] {$\{d_0\}$} (b_1)
	(d_0) edge node[auto] {$\{b_1\}$} (d_1)
	%(c_1) edge node[auto] {$\{b_0\}$, $2$} (c_0)
	(c_0) edge node[auto] {$\{d_1\}$} (c_1)
;

\TState{a_0, b_0, c_0, d_0,e_0}


\end{tikzpicture}

\caption[Example of ABAN]{An example of ABAN}\label{exampleABAN}
\end{figure}

As to reachability problem, given ABAN, joint reachability $REACH (\alpha,\Omega)$ can be formalized as: joint state $\Omega$ is reachable iff there exists a trajectory $t$ s.t. $\alpha\cdot t=\Omega$.
Partial reachability $reach(\alpha,\omega)$ is defined analogously: local state $\omega=a_i$ is reachable iff there exists a trajectory $t$ s.t. $(\alpha\cdot t)[a]=a_i$.
$REACH (\alpha,\Omega)$ and $reach(\alpha,\omega)$ take Boolean values \textbf{True}, \textbf{False} or \textbf{Inconclusive} if it cannot be decided.
In Figure \ref{exampleABAN}, $\Omega=\langle a_1,b_1,c_1,d_1,e_0\rangle$ or $\omega=a_1$ is reachable from initial state $\alpha$ \textit{via} trajectory $t$, \textit{i.e.} $reach(\alpha,a_1)=\textbf{True}$ and $REACH(\alpha,\Omega)=\textbf{True}$. 
In fact, the reachability of a joint state even a global state is equivalent to that of one local state. 

\paragraph{\textbf{One of the advantages}}
Many biological networks are encoded in Boolean style, \textit{e.g.} in \cite{akutsu2007control,kauffman1969}, because BN is a simple formalism but with strong applicability: discretization in BN is a way to handle the imprecision of \textit{a priori} knowledge on the model.
However BN may be not expressive enough.
As to the modeling of the dynamic behavior ``$a\gets$ $1$ at moment $t+1$ if $b=1$ at moment $t$'', one has $a(t+1)=b(t)$ in BN.
$a$ always follows the evolution of $b$ but with a potentially unwanted behavior ``$a\gets 0$ when $b=0$ at moment $t$''.
ABAN models this dynamics as \textit{via} \ac{b_1}{a_0}{a_1} without this redundancy. 
Besides, BNs are transformable to ABANs, and this property makes our approach applicable to a wider domain (Appendix \ref{appendix:trans}).

\subsection{Simplified Local Causality Graph (SLCG)}\label{sec:SLCG}
Paulev\'e \textit{et al.} \cite{pauleve2011} have invented Local Causality Graph (LCG) to analyze reachability problems statically.
LCG abstracts the original problem through an over-approximation (necessary condition) and an under-approximation (sufficient condition).
It is a very efficient tool as there is no global search and all the operations are bounded in polynomial complexity.
However LCG does not guarantee to obtain a result, i.e. some inconclusive instances satisfy the necessary condition but fail sufficient conditions, thus one has no clue about the reachability of these instances.
In this thesis, we make use of the LCG by removing some elements needed only in multivalued networks, naming it SLCG (Simplified LCG), then we try to analyze it more deeply to solve inconclusive cases of binary valued systems.
In fact, Didier \textit{et al.} have shown a technique to transform a multivalued network to Boolean network \cite{didier2011mapping}, which provides us the applicability to multivalued networks.

\begin{definition}[SLCG]\label{defSLCG}
Given an ABAN $AB = (\Sigma,L,T)$, an initial state $\alpha$ and a target state $\omega$, LCG $l= (V_{\mathrm{state}},V_{\mathrm{solution}},E)$ is the smallest recursive structure with $E \subseteq (V_{\mathrm{state}}\times V_{\mathrm{solution}})\cap (V_{\mathrm{solution}}\times V_{\mathrm{state}})$ which satisfies:
\begin{eqnarray*}
    \omega&\in& V_{\mathrm{state}} \\
    a_i\in V_{\mathrm{state}} &\Leftrightarrow& \{ (a_i, sol_{a_i})| a_i\in \alpha\}\subseteq E \\
    sol_{a_i}\in V_{\mathrm{solution}}&\Leftrightarrow& \{ (sol_{a_i},\mathbf{V}_a (sol_{a_i})\}\subseteq E
\end{eqnarray*}
where $V_{\mathrm{state}}\subseteq LS$ is the set of local states, $V_{\mathrm{solution}}\subseteq T$ is the set of solutions and $\mathbf{V}_a$ is the set of required local states of $sol_{a_i}$.
\end{definition}

\begin{remark}
LCGs consist of three kinds of nodes: state nodes corresponding to the states of automata {\rm($V_{\mathrm{state}}$)}, objective nodes corresponding to the state transition paths within one automata {\rm($V_{\mathrm{objective}}$)}, solution nodes corresponding to the transitions to be used for each state transition path {\rm($V_{\mathrm{solution}}$)}.
However, under the circumstance of ABAN, objective nodes are no longer needed.
Because for one state $a_i$ to be reached, the only possible path is $a_{1-i}\to a_i$.
Unlike multi-valued case, for example, if one wants to reach $a_1$ from $a_0$ (suppose possible states for $a$ are 0,1,2), there are in fact infinite possibile paths: $0\to 1,\ 0\to 2 \to 1,\ 0 \to 2 \to 0 \to 1,\ \ldots$
This simplification in fact reduces the searching space and reinforces the conclusiveness.
\end{remark}

\begin{example}
    Figure \ref{LCGexample} shows the SLCG for analyzing $reach(a_1)$ in Figure \ref{exampleABAN}.
    \begin{figure}[ht]
        \centering
        \begin{tikzpicture}[aS,scale=0.7, every node/.style={scale=0.7}]  
  	
  	\startl{a_1};
    \node[Asol,left of=a_1] (a_1s1){};
    \node[Aproc,left of=a_1s1] (e_1){$e_1$};
    \path 
    (a_1s1) edge (e_1)
    (a_1) edge (a_1s1)
    ; 
  	\specl{above}{a_1}{b_1};
  	\link{b_1}{d_0};
  	\edl{d_0};
  	\specl{below}{a_1}{c_1};
	\link{c_1}{d_1};
    \path (d_1s) edge (b_1);
    \end{tikzpicture}

        \caption[SLCG]{Visualization of SLCG, with the squares representing local states and small circles representing solution nodes.
        $\varnothing$ signifies that there is no need to link any transitions, i.e the former state $d_0$ is in the initial state.}
        \label{LCGexample}
    \end{figure}
\end{example}

Algorithm \ref{AlgConstructLCG} describes how to construct an SLCG from an ABAN $AB = (\Sigma,L,T)$.
Starting from  a given target state $Ls=\omega$, one can find all the transitions $T_s\subseteq T$ reaching $\omega$ and add edges $\omega \to T_s$.
Then we find all the heads $A$ of $T_s$ and add edges $T_s \to A$ and replace $Ls$ with $A$ (recursion).
Finally, we update the structure until $Ls\subseteq \alpha$ or there is no transition with body in $Ls$.

Intuitively, when the recursive construction is complete, SLCG is in fact a digraph with state nodes $V_{\mathrm{state}}$ and solution nodes $V_{\mathrm{solution}}$. 
$E$ consists of the edges between local state nodes and solution nodes. 
To access certain local states, at least one of its successor solutions (corresponding transitions from solution nodes) need to be fired; to make one solution node firable, all of its successor local states need to be satisfied. 
A recursive reasoning of reachability begins with a state node representing target local state, goes through $a_i\to sol_{a_i}\to b_j \cdots$ and ends with initial state (possibly reachable) or a local state without solution successor (unreachable). 
With SLCG, it is easy to verify whether their are potential pathways from the target state $\omega$ to the initial state $\alpha$.
If there does not exist such a pathway, one can ensure that $\omega$ is not reachable from $\alpha$.
%\cite{pauleve2012} explains this local reasoning.

\begin{definition}[Pseudo-reachability]\label{defPseudoReach}
Given an SLCG $l=(V_{\mathrm{state}},V_{\mathrm{solution}},E)$ with global initial state $\alpha$, the pseudo-reachability of node $v\in V_{\mathrm{state}}$ is defined as
\begin{equation}
\nonumber
    reach'(\alpha,v)=
    \begin{cases}
        \mathrm{\bf True} & {\rm if\ } v\in \alpha\\
        \mathrm{\bf False} & {\rm if\ } v\not\in \alpha\ {\rm and} \not\exists(s,sol) \in E\\
        \bigvee_{(s,sol) \in E} \mathrm{firable}(sol) & otherwise
        %\bigvee_{(s,sol) \in E}  (\bigwedge_{(sol,s)\in E} reach'(\alpha,s)) & {\rm otherwise}
    \end{cases}
\end{equation}
where $\mathrm{firable}(sol)=\bigwedge_{(sol,s)\in E} reach'(\alpha,s)$. 

\end{definition}

\subsection{Conclusiveness}\label{sec:conclusiveness}
However, pseudo-reachability is named ``pseudo'' because it is only an over-approximation of the reachability, i.e. it verifies a necessary condition of the reachability.
\begin{example}\label{example:unreach}
    In Figure \ref{fig:limitation}, $reach'(l,c_1)=reach'(l,a_1)\land reach'(l,b_1)=reach'(l,a_0)\land reach'(l,b_0)=\textbf{True}$. Both $a_1$ and $b_1$ are reachable, but they can not be reached simultaneously.
    In such LCG, there are two branches, $a_1\to b_0$ and $b_1\to a_0$, the automata $a$ and $b$ involve themselves in different branches, the reachability of $a_1$ impedes the reachability of $b_1$ and \textit{vice versa}.
\end{example}

\begin{figure}[ht]
    \centering
    \begin{minipage}{0.4\textwidth}
\centering
\begin{tikzpicture}[apdotsimple/.style={apdot},scale=0.7, every node/.style={scale=0.8}]
\scriptsize
\TSort{(0,0)}{a}{2}{l}
\TSort{(2,0)}{b}{2}{l}
\TSort{(4.5,0)}{c}{2}{l}

% with delays
\path[local transitions]

	(c_0) edge node[auto] {\{$a_1, b_1$\}} (c_1)
	(a_0) edge node[auto] {\{$b_0$\}} (a_1)
	(b_0) edge node[auto] {\{$a_0$\}} (b_1)
;
\TState{a_0, b_0, c_0}
\end{tikzpicture}
\end{minipage}\hfill
\begin{minipage}{0.6\textwidth}
\centering
\begin{tikzpicture}[aS,scale=0.7, every node/.style={scale=0.7}]  
  	
  	\startl{c_1};
  	\specl{above}{c_1}{a_1};
  	\link{a_1}{b_0};
  	\edl{b_0};
  	\specl{below}{c_1}{b_1};
	\link{b_1}{a_0};
  	\edl{a_0};
\end{tikzpicture}
\end{minipage}
    \caption[Limitation of SLCG 1]{$\Sigma=\{a,b,c\}$, $T=\{\acm{b_0}{a_0}{a_1},\ \acm{a_0}{b_0}{b_1},\ \acm{a_1,b_1}{c_0}{c_1}\},\omega=c_1$}
    \label{fig:limitation}
\end{figure}

Also, the recursive reasoning does not terminate if there exists cycles in SLCG. 
While computing the pseudo-reachability, self-dependent structure  $reach'(l,a_i)=\ldots=reach'(l,a_i)$ might appear and cannot be computed by using Definition \ref{defPseudoReach}. 
In Figure \ref{fig:limitation2}, 
Dealing with cycles becomes inevitable.

\begin{figure}[ht]
    \centering
    \begin{tikzpicture}[aS,scale=0.9, every node/.style={scale=0.9}]  
  	
  	\startl{c_1};
  	\specl{above}{c_1}{a_1};
  	\link{a_1}{b_0};
  	\edl{b_0};
  	\specl{below}{c_1}{b_1};
	\link{b_1}{a_0};
  	\edl{a_0};
    \path
    	(a_0s) edge (a_1)
        (b_0s) edge (b_1)
    ;
    \end{tikzpicture}

    \caption[Limitation of SLCG 2]{SLCG with cycles, $\Sigma=\{a,b,c\}$, $T=\{\acm{b_0}{a_0}{a_1},\ \acm{a_0}{b_0}{b_1},\ \acm{a_1,b_1}{c_0}{c_1},\varnothing \to a_0, \varnothing \to b_0\},\omega=c_1$}
    \label{fig:limitation2}
\end{figure}

\section{Preprocessing of Simplified Local Causality Graph}\label{sec:chap3preprocessing}
\subsection{Detection and Removal of Cycles}\label{sec:cycles}
\begin{definition}[Cycle]
In an SLCG, a cycle is formed by a sequence of nodes linked as follows: $a_i\to \circ \to \cdots \to \circ \to a_i$, where circles stand for solution nodes.
\end{definition}

To identify cycles, we search instead Strongly Connected Components (SCC) of size greater than one.
Because cycles may interlace and there is no such problem in SCC.
In other words, a SCC may contain several nested cycles which connect to each other.
\cite{tarjan1972} shows that the detection of SCCs can be done in $O (|V|+|E|)$ time, with $|V|$ the cardinality of the vertices and $|E|$ the cardinality of the edges.
SLCG is usually a sparse graph, as in biological systems, the components mostly interact with only a part of the system, hence the out-degree can be considered of $O (1)$ and the detection of SCCs\footnote{Implementation in Python3 at \url{https://github.com/alviano/python/blob/master/rewrite_aggregates/scc.py}} can be done in $O(|V|)$, \textit{i.e.} linear time.

\begin{theorem}\label{th:break_cycle}
Given a cycle $x\to \circ \to \cdots \to \circ \to x$ in an SLCG, if there is at most one incoming edge to the cycle, the cycle can be removed.
\end{theorem}

\begin{proof}
If there is no incoming edge, the target state $y$ must be in the cycle. 
The edge $y.pred\to\circ\to y$ can be removed, because the reachability of $y.pred$ requires $y$, but $y$ is the target state, which is never reached before the other local states in the SLCG are reached.
Thus the transition corresponding to this edge is never fired and the edge can be removed.
Similarly, if there is an outside incoming edge $a\to \circ \to x$, $a$ must be the successor of target state $y$ or the target itself, $x.pred\to\circ\to x$ can hence be removed.
\end{proof}

\begin{example}
    \begin{figure}[ht]
        \centering
        \begin{tikzpicture}[->, >=stealth]
\scriptsize
\def \n {6}
\def \radius {1.4cm}
\def \margin {8} % margin in angles, depends on the radius



\node[draw] at ({360/\n * 2}:\radius) (x) {$x$};
\node[draw, circle, minimum size=4pt, inner sep = 0] at ({360/\n * 3}:\radius) {};
\node[draw] at ({360/\n * 4}:\radius) (y) {$y$};
\node[draw, circle, minimum size=4pt, inner sep = 0] at ({360/\n * 5}:\radius) {};
\node[draw] at ({360/\n * 6}:\radius) (z) {$z$};
\node[draw, circle, minimum size=4pt, inner sep = 0] at ({360/\n * 1}:\radius) {};
\foreach \s in {3,...,\n}
{
  
  \draw ({360/\n * (\s - 1)+\margin}:\radius) 
    arc ({360/\n * (\s - 1)+\margin}:{360/\n * (\s)-\margin}:\radius);
}
\foreach \s in {1,2}
{
  
  \draw[dashed] ({360/\n * (\s - 1)+\margin}:\radius) 
    arc ({360/\n * (\s - 1)+\margin}:{360/\n * (\s)-\margin}:\radius);
}
\node[left = 0.8cm of x, draw, circle, minimum size=4pt, inner sep = 0] (sol) {};
\node[left = 0.8cm of sol, draw] (origin) {$a$};
\draw (origin) to [bend left] (sol);
\draw (sol) to [bend left] (x);

\node[right = 0.8cm of z, draw, circle, minimum size=4pt, inner sep = 0] (solz) {};
\node[right = 0.8cm of solz, draw] (ext) {$w$};
\node[right = 0.8cm of ext] (etc) {$\cdots$};
\draw (z) to [bend left] (solz);
\draw (solz) to [bend left] (ext);
\draw (ext) to [bend left] (etc);
\end{tikzpicture}
        \caption[SLCG with cycles]{SLCG $l$ containing cycle $x\to \circ \to y \to \circ \to z\to \circ \to x$}
        \label{cycle1}
    \end{figure}
    
    In Figure \ref{cycle1}, the pseudo-reachability of $a$ is 
    $$reach'(l,a)=reach'(l,x)=reach'(l,y)=reach'(l,z)=reach'(l,x)\lor reach'(l,w)$$
    To reach $x$, we need to reach $z$, but $z$ cannot depend on $x$ as $x$ is already to be reached. 
    Self-dependence appears: $x$ is reachable if $x$ is reachable.
    Thus edge $z\to \circ \to x$ is deleted (dashed line).
\end{example}
Unfortunately, not all cycles are removable via Theorem \ref{th:break_cycle}. Example \ref{example:cycles} explains the issue.

\begin{example}\label{example:cycles}
    \begin{figure}[ht]
        \centering
        \begin{tikzpicture}
\scriptsize
\def \n {6}
\def \radius {1.4cm}
\def \margin {8} % margin in angles, depends on the radius



\node[draw] at ({360/\n * 2}:\radius) (x) {$x$};
\node[draw, circle, minimum size=4pt, inner sep = 0] at ({360/\n * 3}:\radius) {};
\node[draw] at ({360/\n * 4}:\radius) (y) {$y$};
\node[draw, circle, minimum size=4pt, inner sep = 0] at ({360/\n * 5}:\radius) {};
\node[draw] at ({360/\n * 6}:\radius) (z) {$z$};
\node[draw, circle, minimum size=4pt, inner sep = 0] at ({360/\n * 1}:\radius) {};
\foreach \s in {1,...,\n}
{
  
  \draw[->, >=latex] ({360/\n * (\s - 1)+\margin}:\radius) 
    arc ({360/\n * (\s - 1)+\margin}:{360/\n * (\s)-\margin}:\radius);
}

\node[left = 0.8cm of x, draw, circle, minimum size=4pt, inner sep = 0] (sol) {};
\node[left = 0.8cm of sol, draw] (origin) {$a$};
\draw[->, >=latex] (origin) to [bend left] (sol);
\draw[->, >=latex] (sol) to [bend left] (x);
\node[below = 0.8cm of origin, draw, circle, minimum size=4pt, inner sep = 0] (soly2) {};
\draw[->, >=latex] (origin) to [bend right] (soly2);
\draw[->, >=latex] (soly2) to [bend right] (y);
\draw[->, >=latex] (sol) to [bend right= 45] (z);
\end{tikzpicture}
        \caption[Removal of cycles]{$x,y,z$ all have external links, thus none of the links can be discarded}
        \label{cycle3}
    \end{figure}
    
    Cycle $x\to \circ \to y \to \circ \to z\to \circ \to x$ is unbreakable according to Theorem \ref{th:break_cycle}, is possesses 3 incoming edges.
    But with Theorem \ref{th:break_cycle2}, if \textbf{OR gates} are removed, the cycle can be dealt with.
    The removal of \textbf{OR gates} is stated in the next section.
%    If in one cycle, multiple state nodes are required by an external solution node, like in Figure \ref{cycle3}, cycle $x\to \circ \to y \to \circ \to z\to \circ \to x$ is unbreakable, thus it is a unsolvable case. 
%    From experimental view, this issue never happens to real models to our knowledge but it is an incapability of our approach impeding completeness.
\end{example}

\begin{theorem}\label{th:break_cycle2}
Given a cycle, if it contains no \textbf{OR gate}% towards outside of cycle
, all the local states in the cycles are unreachable.
\end{theorem}

\begin{proof}
Suppose an arbitrary cycle $C=a_i\to \cdots b_j\to\cdots \to a_i$, with $\to$ an edge in the LCG.
Note that $reach'(\alpha,a_i)\implies reach'(\alpha,b_j)\implies reach'(\alpha,b_j.next)\implies \cdots\implies reach'(\alpha,a_i)$.
According to the definition of $reach'$, $reach'(\alpha,a)=\mathbf{True}$ only if $\exists c_k\in C$ and $c_k\in \alpha$.
If there exists such $c_k$, $C$ should not exist as the reasoning stops at $c_k$ and does not form a cycle, contradiction.
$reach'(\alpha,a_i)=reach'(\alpha,b_j)=\cdots =\mathbf{False}$.
\end{proof}

\subsection{Decomposition of SLCG}
As every \textbf{OR gate} has multiple choices, to avoid combinatorial explosion, we use a simple heuristic: 
choose randomly one assignment for each trial.
Then we can construct a new LCG without \textbf{OR gate}, every state node has exactly one successor solution node, see Figure \ref{fig:heuristics}.

\begin{figure}[ht]
    \centering
    \begin{tikzpicture}[level/.style={sibling distance=70mm/#1},scale=0.5]
\node [circle,draw] (z){$t$}
  child {node [circle,draw,fill=gray] (a) {}
    child {node [circle,draw,fill=gray] (b) {}
      child {node {$\vdots$}
        child {node [circle,draw] (d) {}}
        child {node [circle,draw,fill=gray] (e) {}}
      } 
      child {node {$\vdots$}}
    }
    child {node [circle,draw] (g) {}
      child {node {$\vdots$}}
      child {node (mid) {$\vdots$}}
    }
  }
  child {node [circle,draw] (j) {}
    child {node [circle,draw] (k) {}
      child {node {$\vdots$}}
      child {node {$\vdots$}}
    }
  child {node [circle,draw] (l) {}
    child {node {$\vdots$}}
    child {node (c){$\vdots$}
      child {node [circle,draw] (o) {}}
      child {node [circle,draw] (p) {}
        child [grow=right] {node (q) {} edge from parent[draw=none]
          child [grow=right] {node (q) {} edge from parent[draw=none]
            child [grow=up] {node (r) {} edge from parent[draw=none]
              child [grow=up] {node (s) {} edge from parent[draw=none]
                child [grow=up] {node (t) {} edge from parent[draw=none]
                  child [grow=up] {node (u) {} edge from parent[draw=none]}
                }
              }
            }
            child [grow=down] {node (v) {}edge from parent[draw=none]}
          }
        }
      }
    }
  }
};
\node [right = 2.3cm of e] (point){$\cdots$};
\path (a) -- (j) node [midway] {};
\path (b) -- (g) node [midway] {};
\path (k) -- (l) node [midway] {};
\path (k) -- (g) node [midway] {};
\path (d) -- (e) node [midway] {};
\path (o) -- (p) node [midway] {};
\path (o) -- (e) node (x) [midway] {}
  child [grow=down] {
    node (y) {}
    edge from parent[draw=none]
  };
\path (q) -- (r) node [midway] {};
\path (s) -- (r) node [midway] {};
\path (s) -- (t) node [midway] {};
\path (s) -- (l) node [midway] {};
\path (t) -- (u) node [midway] {};
\path (z) -- (u) node [midway] {};
\path (j) -- (t) node [midway] {};
\path (y) -- (x) node [midway] {};
\path (v) -- (y)
  node (w) [midway] {};
\path (q) -- (v) node [midway] {};
\path (e) -- (x) node [midway] {};
\path (o) -- (x) node [midway] {};
\path (y) -- (w) node [midway] {};
\path (v) -- (w) node [midway] {};
\path (r) -- (c) node [midway] {};
\end{tikzpicture}
    \caption[Random choice on \textbf{OR gates}]{Random choice on \textbf{OR gates}. Descending from the target state, when we encounter an \textbf{OR gate}, we choose randomly one of its branches. Circles filled gray stand for one possible assignment.}
    \label{fig:heuristics}
\end{figure}

\section{Reachability Analysis}
After preprocessing, we can get rid of cycles and \textbf{OR gates}.
The next step is to analyze an SLCG with only \textbf{AND gates}.
We need to find a trajectory reaching all the components of the \textbf{AND gates} simultaneously.
Here we propose two approaches: Reachability via search in permutations (PermReach) which is a partial search and Reachability via Answer Set Programming (ASPReach) which is a exhaustive search.
\subsection{Reachability via Permutations (PermReach)}

The components of an \textbf{AND gate} form a joint state, and if the joint state is reachable, the corresponding transition of \textbf{AND gate} can be fired. 

In Figure \ref{fig:unreach}, $s=\{ b_1,c_1\}$ is a joint state, when $s$ is reached, transition \ac{b_1,c_1}{a_0}{a_1} is firable.
The reachability of a joint state can be then formulated as sequential reachability:
\begin{definition}[Sequential reachability]
Let $s=\{ls_1,\ldots,ls_n\}$ be a joint state, $p_1,\ldots ,p_n$ be a permutation of $1,\ldots ,n$ and $seq=ls_{p_1}::\ldots::ls_{p_n}$ be a sequence.
The sequential reachability of $seq$ is denoted %$reach(\alpha,seq)=reach(\alpha,ls_{p_1})::\ldots::reach(\alpha,ls_{p_n})$.
$REACH(\alpha,seq)=reach(s_{p_1},ls_{p_1})::\ldots::reach(s_{p_n},ls_{p_n})$, where $s_{p_1}=\alpha$, $s_{p_i}=s_{p_i-1}\cdot t_{p_i-1}$, $t_{p_i-1}$ is the obtained trajectory by the SLCG.
From initial state $\alpha$, $REACH(\alpha,seq)=\mathbf{True}$ if $s$ is reached in the order $seq$.
\end{definition}

\begin{example}\label{example:order}
Figure \ref{fig:unreach} shows the SLCG for the reachability of $c_1$ in ABAN with transitions $\mathbf{T}=\{\acm{a_1,b_1}{c_0}{c_1},\acm{b_0}{a_0}{a_1},\acm{c_0}{b_0}{b_1}\}$.
$a_1$ and $b_1$ are reachable respectively but is not necessarily for the joint state $s=\{a_1,b_1\}$.
If we begin with the branch with $a_1$, $s$ is reachable with trajectory $\acm{b_0}{a_0}{a_1}::\acm{c_0}{b_0}{b_1}::\acm{a_1,b_1}{c_0}{c_1}$. 
However, if we begin with the branch $b_1$, after firing $\acm{c_0}{b_0}{b_1}$, $b_0$ is no longer reachable, resulting the unreachability of $a_1$.
We have $REACH(\alpha,a_1::b_1::c_1)=\mathbf{True}$ and $REACH(\alpha,b_1::a_1::c_1)=\mathbf{False}$.
\end{example}

\begin{figure}[ht]
\centering
\begin{tikzpicture}[aS]  
  	
  	\startl{c_1};
  	\specl{above}{c_1}{a_1};
  	\link{a_1}{b_0};
  	\edl{b_0};
  	\specl{below}{c_1}{b_1};
	\link{b_1}{c_0};
  	\edl{c_0};
    \end{tikzpicture}

\caption[Ordering in SLCG]{The ABAN and the SLCG of \textit{Example \ref{example:order}}, $\alpha=\langle a_0,b_0,c_0\rangle$. 
The only difference with \textit{Example \ref{example:unreach}} is the transition \ac{c_0}{b_0}{b_1}.
%The reachability depends on firing order of transitions
}
\label{fig:unreach}
\end{figure}

As the firing order matters, we come to verify all the possible sequential reachabilities of certain joint state to verify its reachability.

\begin{proposition}\label{theoperm}
Given joint state $s=\{ls_1,\ldots,ls_n\}$, with all the local states in $s$ are reachable: $reach(\alpha,ls_i)=\mathbf{True},\ \forall i\in[1,n]$, the set of permutations of $s$ is denoted $Perm(s)=\{(ls_1::ls_2,::\ldots ::ls_n),\ \cdots,\ (ls_n::ls_{n-1}::\ldots,::ls_1)\}$. $\bigvee_{i\in Perm(s)} REACH(\alpha,i)=\mathbf{True}$ is a sufficient condition of $REACH(\alpha,s)=\mathbf{True}$.
\end{proposition}

\begin{proof}
If $\exists perm_i\in Perm(s)$ s.t. $REACH(\alpha,perm_i)=\mathbf{True}$, $s$ can be reached according to the order in $perm_i$.
To reach $s$, every local state in SLCG is mandatory to be reached, as according to the definition of SLCG, it is the smallest structure which contains all the needed local states and transitions for the target state. As long as there is no \textbf{OR gates}, all the transitions in the SLCG must be fired to reach the target state.
For solvable constraints, $Perm(s)$ can cover some of the admissible orders.
\end{proof}

In case where the successors of certain \textbf{AND gate} contain other \textbf{AND gates}, we cannot directly obtain its reachability because the reachability of the successor \textbf{AND gates} are unknown.
We analyze first the simple \textbf{AND gates} $simp$, i.e. the successors of $simp$ do not contain any \textbf{AND gates}.
If all local states within $simp$ are reachable via the search of permutations, we can update the initial state by firing all the transitions and also update the SLCG by deleting the successors of $simp$. 
Then we restart this process from new simple \textbf{AND gates} until we reach finally the target state.

However the method of PermReach is not complete. 
If there are constraints in different branches, traversing all the permutations may be not sufficient to find admissible trajectories towards the target state.
In Figure \ref{FigConflictInForks}, among the simple \textbf{AND gates}, if $sol_{c_1}$ is solved first, automaton $d$ will be at the state $d_1$, which disables the reachability of $b_1$.
The trajectory towards $a_1$ may not be retrievable by PermReach even if $a_1$ is reachable.
\begin{figure}[ht]
\centering
\begin{tikzpicture}[aS]  
  	
	\startl{a_1};
	\specl{above}{a_1}{b_1};
	\specl{above}{b_1}{d_0};
	\edl{d_0};
	\link{b_1}{a_0};
 	\edl{a_0};
	\link{a_1}{c_1};
 	\link{c_1}{d_1};
 	\link{d_1}{c_0};
 	\edl{c_0};
 	\specl{below}{c_1}{e_0};
 	\edl{e_0};
    \end{tikzpicture}
\caption[Counterexample of PermReach]{Counterexample of PermReach, $\alpha=\langle a_0,b_0,c_0,d_0,e_0\rangle$. 
$reach(\alpha,a_1)$=\textbf{True} but \textbf{Inconclusive} is given by PermReach.
}\label{FigConflictInForks}
\end{figure}
\subsection{Reachability via Answer Set Programming (ASPReach)}
We use ASP  (Answer Set Programming) \cite{baral2003knowledge} to analyze the newly obtained LCG with only \textbf{AND gates}.
ASP is a prolog-like declarative programming paradigm.
It uses description and constraints of the problem (called rule) instead of imperative orders.
ASP solver tackles problems by generating all the possibilities respecting the constraints. 
We use Clingo \cite{gebser2016theory} which is a combination of grounder Gringo and solver Clasp. 
Given an input program with first-order variables, grounder computes an equivalent ground (variable-free) program for an ASP program, while solver selects admissible solutions (answer sets) in the ground.

A rule is in the following form:
$$a_0 \gets a_1 , \ldots , a_m, not\ a_{m+1}, \ldots , not\ a_n.$$
where the element on the left of the arrow is called head and the ones on the right called body.
%All the predicates $a_i$, with $0\leq i \leq n$ are replaceable. 
$a_0$ is \textbf{True} if $a_1 , \ldots , a_m$ are \textbf{True} and $a_{m+1}, \ldots , a_n$ are \textbf{False}.
Some special rules are noteworthy. 
A rule where $m = n = 0$ is called a fact and is useful to represent data because the left-hand atom $a_0$ is thus always \textbf{True}.
It is often written without the central arrow.
On the other hand, a rule where $n > 0$ and $a_0 = \perp$ is called a constraint.
As $\perp$ can never become \textbf{True}, if the right-hand side of a constraint is \textbf{True}, this invalidates the whole solution.
Constraints are thus useful to filter out unwanted solutions.
The symbol $\perp$ is usually omitted in a constraint.

Programs can yield no answer set, one answer set, or several answer sets. 
For example, the program \texttt{b:- not c. c:- not b.}  produces two answer sets: $\{b\}$ and $\{c\}$.
Indeed, the absence of $c$ makes $b$ true, and conversely absence of $b$ makes $c$ true. 
Cardinality constraints are another way to obtain multiple answer sets. 
The most usual way of using a cardinality is in place of $a_0$:
$$l \{q_1, \ldots , q_k \} u \gets a_1, \ldots , a_m, not\ a_{m+1}, \ldots , not\ a_n.$$
where $k \geq 0$, $l$ is an integer and $u$ is an integer or $\infty$. 
Such cardinality means that under the condition that the body is satisfied, the answer set $X$ must contain at least $l$ and at most $u$ atoms from the set $\{q_1, \ldots  , q_m\}$, or, in other words: $l \leq |\{q_1, \ldots  , q_m\} \cap X| \leq u$. %where $\cap$ is the symbol of sets intersection and |A| denotes the cardinality of set A.

After deleting \textbf{OR gates}, to encode the reachability problem in ASP, we first describe the facts:

Predicate \texttt{init(a,i)} shows the automaton $a$ is at initial state $i$. %\texttt{comp(n,a,i)} shows the joint state $a_i$ needed for transition No.$n$. \texttt{transition(n,b,j)} shows the transition No.$n$ allows the automaton $b$ change to state $j$.
Predicate \texttt{node(a,i,n)} shows the node $a_i$ in the LCG is numbered $n$, while \texttt{parent(n1,n2)} expresses node No.$n_1$ is the predecessor of No.$n_2$.
The LCG in Figure \ref{fig:limitation} is encoded as follows:
\begin{Verbatim}[commandchars=\\\{\}]
init(a,0). init(b,0). init(c,0).

node(a,1,1). node(b,1,2). node(c,1,3).
node(b,0,4). node(c,0,5).

parent(1,2). parent(1,3).
parent(2,5). parent(3,4).
\end{Verbatim}
\subsubsection{Reachability Analyzer ASPReach}
In this section, we present our analyzer \textbf{ASPReach}: an algorithm for checking the reachability of a target local state $\omega$ from a global initial state $\alpha$ (which can also be partial) in a given ABAN.
However, exhaustive search leads to heavy computation and huge need of memory.
The algorithm proposed below tries to overcome those shortcomings by combining static analysis and stochastic search into the following hybrid approach.


\paragraph{{\bf ASPReach}:}

\begin{itemize}
    \item Input: An ABAN $AB$, an initial state $\alpha$, a target state $\omega$ and a max number of iterations $k$
    \item Output: $reach(\omega)\in\{\mathbf{False},\mathbf{True},\mathbf{Inconclusive}\}$
\end{itemize}
\begin{enumerate}
    \item Construct the LCG $l=LCG(AB,\alpha,\omega)$
    \item Try to remove all cycles and prune useless edges from $l$
    \item Try to prove unreachability of $\omega$ in $l$ using pseudo-reachability $reach'(l,\omega)$ and return $\mathbf{False}$ if $reach'(l,\omega)=\textbf{False}$
    \item Try at most $k$ times
    \begin{itemize}
    \item $l'\gets l$
    %\item  Transform randomly each OR gate $O$ of $l'$ into simple gate
    \item Simplify each \textbf{OR gate} such that $l'$ is a LCG with only \textbf{AND gates}
    \item If there remain cycles:
        \begin{itemize}
            \item Back to step (iv)
        \end{itemize}
    \item Generate all trajectory that starts with $\alpha$ in $l'$ using ASP
    \begin{itemize}
        \item If a trajectory $t$ ending with $\omega$ is found, return $\mathbf{True}$
    \end{itemize}
    \end{itemize}
    \item return $\mathbf{Inconclusive}$
\end{enumerate}

%\subsubsection*{Recap}
%Now all the subroutines are introduced. 
%Among them, only the heuristics at \textbf{OR Gates} is not complete.
SLCG illustrates the causality between necessary transitions to be fired to reach the target state; 
the tentative of removing cycles simplifies the SLCG and keep the reachability unchanged; 
pseudo-reachability allows one to filter some unreachable cases based on the topology of SLCG. 
The heuristic approach is the core of our algorithm.
Stochastic choices avoid combinatorial explosion on different \textbf{OR Gates}.
The ASP part searches thoroughly the result but does not traverse the whole state space (ASP solver starts from constraints, finds one consistent order and terminate the search). 
%It has to be pointed out that INCONCLUSIVE may appear as an output of the Algorithm \ref{algOverall}.
%If the target state $\omega$ is in fact reachable, usually if there exists multiple assignments which contain an admissible solution.
%Moreover, the bigger $k$ we set, the less probable we miss the good assignments.

\begin{algorithm}[htb]
\begin{algorithmic}[1]
\State Input: SLCG $l=(V_{\mathrm{state}},V_{\mathrm{solution}}, E)$, an integer $k$
\State Output: reachability $r$ and a trajectory $t$
\State Compute SCCs, classify them into $\mathrm{SCC1}(l)$ with at most 1 incoming edge and $\mathrm{SCC2}(l)$ otherwise
\State \textcolor{gray}{// 1) Break all cycles and prune useless branches}\label{delete_cycle_begin}
\For{each $(V'_{\mathrm{state}} \subseteq V_{\mathrm{state}},V'_{\mathrm{solution}} \subseteq V_{\mathrm{solution}}) \in \mathrm{SCC1}(l)$}
    \For{each $v \in V'_{\mathrm{state}}$}
    \If {$\exists(v,v')\in E, v' \in (V_{\mathrm{solution}} \setminus V'_{\mathrm{solution}})$}
        \State $E\gets E\backslash \{(v,v'')|v''\in V'_{\mathrm{solution}},(v,v'')\in E\}$
    \EndIf
\EndFor
\EndFor \label{delete_cycle_end}
\State{\textcolor{gray}{// 2) remove useless nodes/edges}} \label{prune_begin}
\State pruned = \textbf{True}
\While{pruned}
    \State pruned = \textbf{False}
    \For {$v \in V_{\mathrm{state}}$}
        \If{$\not\exists(v,v')\in E$}
            \State $V_{\mathrm{state}} \gets V_{\mathrm{state}}\backslash \{v\}$; $E\gets E\backslash \{ (v'',v)\in E\}$
            \State $E\gets E\backslash \{ (v'',sol)\in E | sol \in \{sol = (A \rightarrow a) \in V_{\mathrm{solution}} | v \in A\}\}$
            \State $V_{\mathrm{solution}} \gets V_{\mathrm{solution}}\backslash \{sol = (A \rightarrow a) \in V_{\mathrm{solution}} | v \in A\}$
            \State pruned = \textbf{True}
        \EndIf
    \EndFor \label{prune_end}
\EndWhile
\State \textcolor{gray}{// 3) Check pseudo-reachability} \label{pseudo_reach_begin}
\If {$pseudoReach(l)=\textbf{False}$}
    \State \Return $(\mathbf{False},\varnothing)$
\EndIf \label{pseudo_reach_end}

\State \textcolor{gray}{// 4) main search loop} \label{main_loop_begin}
\For{each $i$ in $1\ldots k$}
    \State $l'=(V'_{\mathrm{state}}, V'_{\mathrm{solution}},E')\gets(V_{\mathrm{state}}, V_{\mathrm{solution}},E)$ 
    \For{$v \in V'_{state}$} \textcolor{gray}{// Treat each OR gates}
        \State pick a random element $(v,v') \in E'$
        \State $E'\gets E' \backslash  \{(v,v'') \in E'| v''\neq v'\}$ with $\nexists i\in \mathrm{SCC2}(l)$ and $i\in E'$
    \EndFor
    \State $(r,t)\gets\mathrm{ASPsolve}(l')$
    \If{$r=\textbf{True}$}
        \Return{$(\mathbf{True},t)$}
    \EndIf
\EndFor \label{main_loop_end}
\State \Return{$(\mathbf{Inconclusive},\varnothing)$}
\end{algorithmic}
\caption{ASPReach}\label{algOverall}
\end{algorithm}

Algorithm \ref{algOverall} provides the detailed pseudocode of the algorithm taking an SLCG $l$ as input whose detailed construction is given in Algorithm \ref{AlgConstructLCG}.
% Cycle deletion
Lines \ref{delete_cycle_begin}-\ref{delete_cycle_end} delete all cycles with at most one incoming edge.
%Getting rid of cycle prevents self-dependent reachability.
% Pruning
After removing cycles, the SLCG may contain nodes without successor.
Such nodes can be pruned since they do not lead to initial state (Line \ref{prune_begin}-\ref{prune_end}).
This preprocessing reduces the search space of the stochastic search performed in step 4.
% Pseudo reach
Now $l$ is pruned and might be cycle-free.
Static analysis of $l$ can then be used as heuristics to check pseudo-reachability (Definition \ref{defPseudoReach}) in order to detect some unreachability cases (Lines \ref{pseudo_reach_begin}-\ref{pseudo_reach_end}) which may conclude before searching.
SLCG shows the dependencies between local states and transitions. 
A pathway in SLCG suggests a possible trajectory of reaching the target state. 
If $pseudoReach(l,\omega)=\textbf{False}$, we can ensure that $\omega$ is unreachable, as pseudo-reachability checks a necessary condition of reachability.
If $reach'(l,\omega)=\textbf{True}$, static analysis is not sufficient for reachability analysis. 
% St
When static analysis fails, a stochastic search is performed at most $k$ times (line \ref{main_loop_begin}-\ref{main_loop_end}) to find a state sequence from the initial state $\alpha$ to target state $\omega$.
If there remain cycles with multiple incoming edges, according to Theorem \ref{th:break_cycle2}, $\omega$ is unreachable.
The value of $k$ will be discussed later in the evaluation section.
Random choices are made to fix a value for each \textbf{OR gate} of the SLCG allowing to perform a reachability check by generating all possible variable assignment order using ASP.
%The detailed of this operation is given in section \ref{sec:OR}.
%After the removal of cycles and the computation of pseudo reachability, the task remains to find a state sequence from the desired state $\omega$ to initial state. 
Keep in mind that every state node is an \textbf{OR gate}, we have to choose one of its successor solution nodes to access the state. 
A set of \textbf{OR gate} choices is called an \textit{assignment}.

After applying heuristics to delete \textbf{OR gates}, we use ASP  (Answer Set Programming) \cite{baral2003knowledge} to analyze the newly obtained SLCG with only \textbf{AND gates}.
ASP is a prolog-like declarative programming paradigm.
It uses description and constraints of the problem (called rule) instead of imperative orders.
ASP solver tackles problems by generating all the possibilities respecting the constraints. 
We use Clingo\cite{gebser2016theory} which is a combination of grounder Gringo and solver Clasp. 
Given an input program with first-order variables, grounder computes an equivalent ground (variable-free) program for an ASP program, while solver selects admissible solutions (answer sets) in the ground.

A rule is in the following form:
$$a_0 \gets a_1 , \ldots , a_m, not\ a_{m+1}, \ldots , not\ a_n.$$
where the element on the left of the arrow is called head and the ones on the right called body.
%All the predicates $a_i$, with $0\leq i \leq n$ are replaceable. 
$a_0$ is \textbf{True} if $a_1 , \ldots , a_m$ are \textbf{True} and $a_{m+1}, \ldots , a_n$ are \textbf{False}.
Some special rules are noteworthy. 
A rule where $m = n = 0$ is called a fact and is useful to represent data because the left-hand atom $a_0$ is thus always \textbf{True}.
It is often written without the central arrow.
On the other hand, a rule where $n > 0$ and $a_0 = \perp$ is called a constraint.
As $\perp$ can never become \textbf{True}, if the right-hand side of a constraint is \textbf{True}, this invalidates the whole solution.
Constraints are thus useful to filter out unwanted solutions.
The symbol $\perp$ is usually omitted in a constraint.

Programs can yield no answer set, one answer set, or several answer sets. 
For example, the program \texttt{b:- not c. c:- not b.}  produces two answer sets: $\{b\}$ and $\{c\}$.
Indeed, the absence of $c$ makes $b$ true, and conversely absence of $b$ makes $c$ true. 
Cardinality constraints are another way to obtain multiple answer sets. 
The most usual way of using a cardinality is in place of $a_0$:
$$l \{q_1, \ldots , q_k \} u \gets a_1, \ldots , a_m, not\ a_{m+1}, \ldots , not\ a_n.$$
where $k \geq 0$, $l$ is a natural number and $u$ is a natural number or $\infty$. 
Such cardinality means that under the condition that the body is satisfied, the answer set $X$ must contain at least $l$ and at most $u$ atoms from the set $\{q_1, \ldots  , q_m\}$, or, in other words: $l \leq |\{q_1, \ldots  , q_m\} \cap X| \leq u$. %where $\cap$ is the symbol of sets intersection and |A| denotes the cardinality of set A.

\subsubsection*{ASP Encoding}

After deleting \textbf{OR gates}, to encode the reachability problem in ASP, we first describe the facts:

Predicate \texttt{init(a,i)} shows the automaton $a$ is at initial state $i$. %\texttt{comp(n,a,i)} shows the joint state $a_i$ needed for transition No.$n$. \texttt{transition(n,b,j)} shows the transition No.$n$ allows the automaton $b$ change to state $j$.
Predicate \texttt{node(a,i,n)} shows the node $a_i$ in the SLCG is numbered $n$, while \texttt{parent(n1,n2)} expresses node No.$n_1$ is the predecessor of No.$n_2$.
The SLCG in Figure \ref{fig:limitation} is encoded as follows:
\begin{Verbatim}[commandchars=\\\{\}]
init(a,0). init(b,0). init(c,0).

node(a,1,1). node(b,1,2). node(c,1,3).
node(b,0,4). node(c,0,5).

parent(1,2). parent(1,3).
parent(2,5). parent(3,4).
\end{Verbatim}

After the facts, we want the nodes to appear in an order by which we can fire all the transitions sequentially from initial state to target state. 

The rough idea is: If different states of one automaton $a$ appear, e.g. $a_0$ and $a_1$.
One of them must be in initial state (suppose $a_0$).
The transitions with head $a_0$ have to be fired before $a_0$ flipping to $a_1$, otherwise there is no solution node in the SLCG which allows $a_1$ return to $a_0$.
In other words, the predecessor of $a_0$ must appear before $a_1$. \texttt{\textcolor{gray}{Core rule}} describes this constraint.

Predicate \texttt{prior(N1,N2)} signifies node $N_1$ appears earlier than $N_2$ in the resulting state sequence.
\texttt{seq(O,a,i)} shows that state node $a_i$ appears in the O-th place in a trajectory.
\texttt{reachable/unreachable} is the final result of the program.

\begin{Verbatim}[commandchars=\\\{\}]
\textcolor{gray}{%Rule 1, a node appears always earlier than its predecessor}
prior(N1,N2) :- parent(N2,N1).
\textcolor{gray}{%Rule 2, transitivity}
prior(N1,N3) :- prior(N1,N2), prior(N2,N3).
\textcolor{gray}{%Rule 3, Core rule}
prior(N1,N2) :- node(P1,S1,N1), node(P2,S2,N2), node(P2,S3,N3), 
                parent(N1,N3), init(P2,S3), S2!=S3, P1!=P2. 
\textcolor{gray}{%target is unreachable if there is a conflict in order}
unreachable :- prior(N1,N2), prior(N2,N1), N1<N2.
\textcolor{gray}{%One node appears once and at least once in a sequence}
1{seq(1..O,P,S)}1 :- O={node(P1,S1,N1):node(P1,S1,N1)},
                     node(P,S,N), not unreachable.
\textcolor{gray}{%Nodes in the sequence are consistent with the order}
:- prior(N1,N2), node(P1,S1,N1), node(P2,S2,N2),
   seq(O1,P1,S1), seq(O2,P2,S2), O1>O2.
\textcolor{gray}{%One place in the sequence cannot be taken by multiple nodes}
:- seq(O1,P1,S1), seq(O2,P2,S2), P1!=P2, O1=O2.
:- seq(O1,P1,S1), seq(O2,P2,S2), S1!=S2, O1=O2.
\textcolor{gray}{%--------output formatting, displaying initial states first}
:- seq(O1,P1,S1), seq(O2,P2,S2), init(P1,S1),
   not init(P2,S2), O1>O2.
:- seq(O1,P1,S1), seq(O2,P2,S2), init(P1,S1), init(P2,S2),
   P1<P2, O1>O2.
reachable :- not unreachable.
\end{Verbatim}

Notation: $a\rhd b$ means $a$ appears before $b$.

When analyzing the SLCG in Figure \ref{fig:limitation},
Rule 1 gives $b_0\rhd a_1$, $a_1\rhd c_1$, $a_0\rhd b_1$, $b_1\rhd c_1$; Rule 2 gives $a_0\rhd c_1$ and $b_0\rhd c_1$; Rule 3 gives $a_1\rhd b_1$ and $b_1\rhd a_1$ which is impossible, therefore there does not exist a state sequence to reach $c_1$ from initial state.
$c_1$ is unreachable.

If we find a state sequence consistent with all the order constraints, we can obtain its corresponding trajectory, thus we are sure that the target state is reachable.

\begin{figure}[ht]
    \centering
    \begin{tikzpicture}[aS]  
  	
  	\startl{a_1};
    \node[Asol,left of=a_1] (a_1s1){};
    \node[Aproc, above left = 0.6cm of a_1s1] (d_1){$d_1$};
    \node[Aproc, below left = 0.6cm of a_1s1] (e_1){$e_1$};
    \node[Asol, left of = d_1](d_1s){};
    \node[Asol, left of = e_1](e_1s){};
    \node[Aproc, left = 0.6cm of e_1s] (d_0){$d_0$};
    \node[Aproc, left = 0.6cm of d_1s] (e_0){$e_0$};
    \node[Asol, left of = d_0](d_0s){};
    \node[Asol, left of = e_0](e_0s){};
    \node[Assol, left = 0.6cm of d_0s](d_0st){$\varnothing$};
    \node[Assol, left = 0.6cm of e_0s](e_0st){$\varnothing$};
    \node[right = 0.1cm of a_1s] {AND};
    \node[left = 0.1cm of a_1s1] {AND};
    \node[above = 0.1cm of a_1] {OR};
    \path 
    (a_1s1) edge (d_1)
    (a_1s1) edge (e_1)
    (d_1) edge (d_1s)
    (e_1) edge (e_1s)
    (d_1s) edge (e_0)
    (e_1s) edge (d_0)
    (d_0) edge (d_0s)
    (e_0) edge (e_0s)
    (d_0s) edge (d_0st)
    (e_0s) edge (e_0st)
    (a_1) edge (a_1s1)
    ; 
  	\specl{above}{a_1}{b_1};
  	\link{b_1}{c_0};
  	\node[Asol, right of = c_0](c_0s){};
    \node[Assol, right = 0.6cm of c_0s](c_0st){$\varnothing$};
  	%\edl{c_0};
  	\specl{below}{a_1}{c_1};
	\link{c_1}{b_0};
	\node[Asol, right of = b_0](b_0s){};
    \node[Assol, right = 0.6cm of b_0s](b_0st){$\varnothing$};
  	%\edl{b_0};
  	\path
  	(b_0s) edge (b_0st)
  	(c_0s) edge (c_0st)
  	;
\end{tikzpicture}

    \caption[Counterexample of ASPReach]{If an SLCG contains such structure, the result could be inconclusive.
    However the inconclusiveness requires $a_1$ does not possess other reachable branches.}
    \label{fig:lcgInconc}
\end{figure}
Still, ASPReach is not complete.
A counter-example is shown in Figure \ref{fig:lcgInconc}, when there are multiple branches of one \textbf{OR gate} leading to unreachability, the result can be inconclusive.
There is a tricky way to deal with this issue when $\mathbf{|OR gates|}$ is not big: we set a limit $n$, if $\mathbf{|OR gates|}<n$, we shift the heuristics on the assignment of \textbf{OR gates} to the enumeration of all possible assignments.
This ``hacking'' can deal with some inconclusiveness.
In the benchmarks in the next section, inconclusive instances appear neither in biological examples nor in random generated tests.

We also show some algorithmic properties of ASPReach:

\begin{theorem}[The termination and correctness of ASPReach]

    Let $l=(V_{\mathrm{state}}, V_{\mathrm{solution}}, E)$ be an SLCG with initial state $\alpha$ and target local state $\omega$ and $k > 0$ be an integer.
    The call \textit{ASPReach(l,k)} terminates.\\
    $ASPReach(l,k)$=$(\mathbf{False},\varnothing)$ only if $\nexists t$ a trajectory in $l$ from $\alpha$ to $\omega$.\\
    $ASPReach(l,k)$=$(\mathbf{True},t)$ only if $\exists t$ a trajectory in $l$ from $\alpha$ to $\omega$.
    The proof is given in appendix \ref{sec:proof}.
\end{theorem}

\begin{theorem}[The complexity of ASPReach]
    Let $l=(V_{\mathrm{state}}, V_{\mathrm{solution}}, E)$ be an SLCG with initial state $\alpha$ and $k > 0$ be an integer.
    Let $s=|V_{\mathrm{solution}}|$ be the number of target state of $l$.
    Let $v = |V_{\mathrm{state}}|$ be the number of vertices of $l$.
    Let $e=|E|$ be the number of edges of $l$.
    The complexity of $\textit{ASPReach(l,k)}$ is $O(v + e + v/2 \times v \times e \times s + v^{2} \times e + v \times e + k \times (v \times e^{2} + 2^{v}))$ which is bound by $O(k \times 2^{v})$.
    Proof is given in Appendix \ref{sec:proof}.
\end{theorem}
\section{R\'esum\'e}
In this chapter, we formally (re)define the modeling framework used for static analysis: Asynchronous Binary Automata Network and some terminology.
Then we dug into the details of static analysis, figured out why they are not conclusive under certain constraints.
To get rid of these constraints, we carry first preprocessing (Section \ref{sec:chap3preprocessing}) to detect and try to delete cycles.
With the preprocessed ABAN, we introduce two reachability analyzers based on SLCG: PermReach and ASPReach.

PermReach relies on a complete search on the permutations of \textbf{AND gates}.
However, permutations do not cover all the possible trajectories but PermReach is very efficient .

ASPReach does a finer work than PermReach, searching all the possible trajectories of a preprocessed ABAN (without cycles and \textbf{OR gates}).

The experimental results are in Chapter \ref{chap:test} Benchmark showing these two approaches can deal with more problems than pure static analysis.

Like static analysis, those two analyzers are not theoretically conclusive.
However, if one wants total conclusiveness, he will need a complete search over state space like exact model checkers do, resulting in state space explosion.